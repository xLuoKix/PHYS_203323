%%%%%%%%%%%%%%%%%%%%%%%%%%%%%%%%%%%%%%%%%%%%%%%%%%%%%%%%%%%%
%%%%%%%%%%%%%%%%%%%%%%%%%%%%%%%%%%%%%%%%%%%%%%%%%%%%%%%%%%%%
%%%%%%%%%%%%%%%%%%%%%%%%%%%%%%%%%%%%%%%%%%%%%%%%%%%%%%%%%%%%
%%%%%%%%%%%%%%%%%%%%%%%%%%%%%%%%%%%%%%%%%%%%%%%%%%%%%%%%%%%%
%%%%%%%%%%%%%%%%%%%%%%%%%%%%%%%%%%%%%%%%%%%%%%%%%%%%%%%%%%%%
\documentclass[12pt]{article}
\usepackage{fancyhdr}
\usepackage{pslatex}
\usepackage{epsfig}
\usepackage{times}
\usepackage{amsmath}
\usepackage{mathrsfs}
\usepackage{textgreek}
\usepackage{setspace}
\usepackage[dvipsnames]{xcolor}
\usepackage[hidelinks]{hyperref}%renewcommand{\topfraction}{1.0}
\renewcommand{\topfraction}{1.0}
\renewcommand{\bottomfraction}{1.0}
\renewcommand{\textfraction}{0.0}
\setlength {\textwidth}{6.6in}
\hoffset=-1.0in
\oddsidemargin=1.00in
\marginparsep=0.0in
\marginparwidth=0.0in                                                                               
\setlength {\textheight}{9.0in}
\voffset=-1.00in
\topmargin=1.0in
\headheight=0.0in
\headsep=0.00in
\footskip=0.50in                                         
\setcounter{page}{1}
\onehalfspacing
\begin{document}
\def\pos{\medskip\quad}
\def\subpos{\smallskip \qquad}
\newfont{\nice}{cmr12 scaled 1250}
\newfont{\name}{cmr12 scaled 1080}
\newfont{\swell}{cmbx12 scaled 800}
%%%%%%%%%%%%%%%%%%%%%%%%%%%%%%%%%%%%%%%%%%%%%%%%%%%%%%%%%%%%
%     DO NOT CHANGE ANYTHING ABOVE THIS LINE
%%%%%%%%%%%%%%%%%%%%%%%%%%%%%%%%%%%%%%%%%%%%%%%%%%%%%%%%%%%%
%     DO NOT CHANGE ANYTHING ABOVE THIS LINE
%%%%%%%%%%%%%%%%%%%%%%%%%%%%%%%%%%%%%%%%%%%%%%%%%%%%%%%%%%%%
%     DO NOT CHANGE ANYTHING ABOVE THIS LINE
%%%%%%%%%%%%%%%%%%%%%%%%%%%%%%%%%%%%%%%%%%%%%%%%%%%%%%%%%%%%

\begin{center}
{\Large
PHYS 20323/60323: Fall 2025 - LaTeX Example
}\\
\end{center}

\noindent\begin{enumerate}
\item At time t = 0 a particle is represented by the wave function 

    
\begin{equation}
\Psi(x)=\begin{cases}
        A\frac{x}{a}, & 0\leq x \leq a \\
        A\frac{(b-x)}{(b-a)}, & a\leq x \leq b \\
        0, & otherwise \\
\end{cases}
\end{equation}
where $A$, $a$, and $b$ are constants. 
\begin{enumerate}
\item (3.3 points) Normalize $\Psi$\ (i.e., Find $A$ in terms of $a$ and $b$).
\item (3.3 points) Where is the particle likely to be found at $t$ = 0?.
\item (3.4 points) What is the expectation value of x?. 
\end{enumerate}
{\item\space\bf The following questions refer to stars in the table bellow} \\
$Note$: $there$ $may$ $be$ $multiple$ $answers$. \\
\begin{tabular}{|c|c|c|c|c|c|c|c|}\hline
Name   &  Mass  &  Luminosity  & Lifetime & temperature & Radius & Variable? \\\hline
$\delta$ Scu.   & 2.0 $M_{\odot}$  &  & $5.0\times10^8$ years &  &  2.0 $R_{\odot}$ & Y \\\hline
$\gamma$ Del.   & 0.7 $M_{\odot}$ &   & $4.5\times$10$^{10}$ years & 5000 K & & N  \\\hline
$\beta$ Cyg. & 1.3 $M_{\odot}$ & 3.5 $L_{\odot}$ & & & & Y \\\hline
$\eta$ Car. & 60. $M_{\odot}$ & 10$^6$ $L_{\odot}$ & $8.0\times10^5$ years & & & Y \\\hline
$\epsilon$ Eri. & 6.0 $M_{\odot}$ & 10$^3$ $L_{\odot}$ & & 20,000 K & & N \\\hline
$\alpha$ Cen. & 1.0 $M_{\odot}$ & & & 6000 K & 1.0 $R_{\odot}$ & N \\\hline
\end{tabular}
\vskip0.15in


\begin{enumerate}
\item (4 points) Which of these stars will produce a planetary nebula. 
\item (4 points) Elements heavier than \texttt{Carbon} will be produced in which stars.
\end{enumerate}
\vskip0.15in

\item \space An electron is found in the spin state (in the $z$-basis):
$\chi=  A\space\left(\begin{array}{cc}
     3i  \\
     4 
\end{array}
\right)$
\vskip0.15in
\begin{enumerate}
\item  (5 points) Determine the possible values of A such that the state is normalized.
\item (5 points) Find the expectation values of the operators \textcolor{red}{$S_x$}, \textcolor{purple}{$S_y$}, \textcolor{orange}{$S_z$} and $\vec{S}^2$.
\end{enumerate}
\indent The matrix representations in the $z$-basis for the components for the electron spin operators are given by: \\\vskip0.1in
\textcolor{red}{$S_x = \frac{\hbar}{2} \left(\begin{array}{cc}
     0 & 1  \\
     1 & 0 
\end{array}
\right)}$
\textcolor{purple}{;}
$\phantom{3844}$\textcolor{purple}{$S_y = \frac{\hbar}{2} \left(\begin{array}{cc}
     0 & -i  \\
     i & 0 
\end{array}
\right)}$}
\textcolor{orange}{;
$\phantom{3844}$ $S_y = \frac{\hbar}{2} \left(\begin{array}{cc}
     0 & -i  \\
     i & 0 
\end{array}
\right)}$
\end{enumerate}
\end{document}
